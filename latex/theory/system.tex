\label{cha:physics}


The model treated is a variation of the conventional Ising model in two dimensions with nearest neighbor interactions 
(NN-interactions). In this work, the next nearest neighbor interactions (NNN-interactions) are additionally considered. 
Therefore, the Hamiltonian is given by 
\begin{align}
    \mathcal{H}(\bm{\sigma})=-J_1\sum_{\langle ij \rangle}\sigma_i\sigma_j-J_2\sum_{[ij]}\sigma_i\sigma_j,
    \label{align:Hamiltonian}
\end{align}
where $\bm{\sigma}\!\in\!\{-\!1,1\}^N$ denotes the microstate with $N$ spins, $\sigma_i$ the components of $\bm{\sigma}$, 
$J_1\!>\!0,J_2\!<\!0$ are coupling constants, $\langle ij \rangle$ denotes the NN-interactions and $[ij]$ the NNN-interactions. The 
studies in this thesis are limited to square lattices with periodic boundary conditions, therefore the total number of spins in a system with 
the lattice length $L$ is $N\!=\!L^2$. 


\section{Ground-state properties}
\label{sec:GroundState}

During the discussion of the ground-state properties of the system described above, Timmons et al. refer in their work~\cite{Timmons2018} to a paper 
of Fan and Wu~\cite{Fan1970} from 1969, containing quite a general analysis of ground-state properties of the vertex model for different 
cases. They mention that for $J_2\in(-\!J_1/2,0]$, the system has a ferromagnetic ground-state, i.e., that all $\sigma_i$ are  either $+\!1$ or $-\!1$. 
To determine the ground-state energy consider the left lattice in Figure~\ref{fig:system1}. 
\begin{figure}[!h]
    \centering
        

\begin{tikzpicture}
        \matrix[
            matrix of math nodes,
            matrix anchor=west,
            row sep=2ex,
            column sep=2ex,
            ] (m1) at (0,0)
            {
                \upF\doE & \upF\doE & \upF\doE & \upF\doE \\
                \upF\doE & \upF\doE & \upF\doE & \upF\doE \\
                \upF\doE & \upF\doE & \upF\doE & \upF\doE \\
                \upF\doE & \upF\doE & \upF\doE & \upF\doE \\                        
            };
            \path[draw=black]   (m1-1-1) edge[dashed]    (m1-2-2)
                                (m1-3-1) edge[dashed]    (m1-2-2)
                                (m1-2-1) edge            (m1-2-2)
                                (m1-1-2) edge            (m1-2-2)                                                        
            ;
            \path[draw=red]     (m1-1-3) edge[dashed]    (m1-2-2)
                                (m1-3-3) edge[dashed]    (m1-2-2)
                                (m1-2-3) edge            (m1-2-2)
                                (m1-3-2) edge            (m1-2-2)                                                                
            ;

            \matrix[
                matrix of math nodes,
                matrix anchor=west,
                row sep=2ex,
                column sep=2ex,
                ] (m2) at (4,0)
                {
                    \upF\doE & \upF\doE & \upF\doE & \upF\doE \\
                    \doF\upE & \doF\upE & \doF\upE & \doF\upE \\
                    \upF\doE & \upF\doE & \upF\doE & \upF\doE \\
                    \doF\upE & \doF\upE & \doF\upE & \doF\upE \\                         
                };
                \path[draw=black]   (m2-1-1) edge[dashed]    (m2-2-2)
                                    (m2-3-1) edge[dashed]    (m2-2-2)
                                    (m2-2-1) edge            (m2-2-2)
                                    (m2-1-2) edge            (m2-2-2)                                                        
                ;
                \path[draw=red]     (m2-1-3) edge[dashed]    (m2-2-2)
                                    (m2-3-3) edge[dashed]    (m2-2-2)
                                    (m2-2-3) edge            (m2-2-2)
                                    (m2-3-2) edge            (m2-2-2)
                ; 
                \matrix[
                    matrix of math nodes,
                    matrix anchor=west,
                    row sep=2ex,
                    column sep=2ex,
                    ] (m3) at (8,0)
                    {
                    \upF\doE & \doF\upE & \upF\doE & \doF\upE \\
                    \upF\doE & \doF\upE & \upF\doE & \doF\upE \\
                    \upF\doE & \doF\upE & \upF\doE & \doF\upE \\
                    \upF\doE & \doF\upE & \upF\doE & \doF\upE \\                         
                    };
                    \path[draw=black]   (m3-1-1) edge[dashed]    (m3-2-2)
                                        (m3-3-1) edge[dashed]    (m3-2-2)
                                        (m3-2-1) edge            (m3-2-2)
                                        (m3-1-2) edge            (m3-2-2)                                                        
                    ;
                    \path[draw=red]     (m3-1-3) edge[dashed]    (m3-2-2)
                                        (m3-3-3) edge[dashed]    (m3-2-2)
                                        (m3-2-3) edge            (m3-2-2)
                                        (m3-3-2) edge            (m3-2-2)
                    ;   
\end{tikzpicture}
    \caption{ 
                The sketch shows the possible ground-state patterns (left) for $J_2\in(-\!J_1/2,0]$, (middle) and (right) for 
                $J_2\in(-\!\infty,-\!J_1/2)$. The solid lines represent the NN-interactions and the dashed ones the 
                NNN-interactions. 
            }
    \label{fig:system1}
\end{figure}
Each bond drawn contributes $\pm1$ to the sums in the Hamiltonian~\eqref{align:Hamiltonian}, they all are $+\!1$ in the case depicted. To avoid double
summations, sum only over the bonds drawn red for each lattice site. Therefore, one obtains the ground-state energy through a 
summation over the whole lattice $$E_0=\sum_{i=1}^{N}\left( -J_1\cdot2-J_2\cdot2 \right)=-2N(J_1+J_2).$$
For $J_2\in(-\!\infty,-\!J_1/2)$, the system has a super-antiferromagnetic ground-state~\cite{Timmons2018}. Thus, the spins order in
striped patterns of alternating spin values, shown in Figure~\ref{fig:system1} (middle and right lattice). The ground-state energy 
of these four possible states can be computed analogously to the previous case, with the difference that the NN-interactions cancel out for 
each lattice site. Hence, the ground-state energy is given by $$E_0=\sum_{i=1}^{N}\left( -J_1\cdot(1-1)-J_2\cdot(-2) \right)=2NJ_2.$$

Comparing these two ground-state energies, one obtains a special case for $J_2=-\!\frac{J_1}{2}$ because then, the two energy-levels 
coincide. This value of $J_2$ separates these two ground-state regimes, called in the following critical point. At the critical point, the system 
has more than the six ground-states of Figure~\ref{fig:system1}. To obtain the other ground-states, regard
the sketches in Figure~\ref{fig:system2}. Flipping the highlighted lines of aligned spins in the sketched environment, the energy of the 
system does not change.   
\begin{figure}[!h]
    \centering
        

\begin{tikzpicture}
    \matrix[
        matrix of math nodes,
        matrix anchor=west,
        row sep=2ex,
        column sep=2ex,
        ] (m1) at (0,0)
        {
            \upF\doE & \upF\doE & \upF\doE & \dots & \upF\doE \\
            \doF\upE & \doF\upE & \doF\upE & \dots & \doF\upE \\
            \upF\doE & \upF\doE & \upF\doE & \dots & \upF\doE \\
                &     & \downarrow & \\ 
            \upF\doE & \upF\doE & \upF\doE & \dots & \upF\doE \\
            \upF\doE & \upF\doE & \upF\doE & \dots & \upF\doE \\
            \upF\doE & \upF\doE & \upF\doE & \dots & \upF\doE \\                  
        };
    \matrix[
        matrix of math nodes,
        matrix anchor=west,
        row sep=2ex,
        column sep=2ex,
        ] (m2) at (5,0)
        {
            \doF\upE & \doF\upE & \doF\upE & \dots & \doF\upE \\
            \doF\upE & \doF\upE & \doF\upE & \dots & \doF\upE \\
            \upF\doE & \upF\doE & \upF\doE & \dots & \upF\doE \\
                &     & \downarrow & \\ 
            \doF\upE & \doF\upE & \doF\upE & \dots & \doF\upE \\
            \upF\doE & \upF\doE & \upF\doE & \dots & \upF\doE \\
            \upF\doE & \upF\doE & \upF\doE & \dots & \upF\doE \\                    
        };
        \begin{scope}[on background layer]
            \node[fit=(m1-2-1)(m1-2-5), fill=white!70!black, rounded corners] {};
            \node[fit=(m1-6-1)(m1-6-5), fill=white!70!black, rounded corners] {};
            \node[fit=(m2-2-1)(m2-2-5), fill=white!70!black, rounded corners] {};
            \node[fit=(m2-6-1)(m2-6-5), fill=white!70!black, rounded corners] {};            
        \end{scope}                                                             
        ; 
\end{tikzpicture}
    \caption{ 
                The sections of horizontal ground-state patterns for the system at the critical point are depicted. The highlighted rows correspond 
                to the lines that are getting flipped.  
            }
    \label{fig:system2}
\end{figure}
To see that, let $E_{\mathrm{new}/\mathrm{old}}$ be the part of the energy, the highlighted line contributes to. For the left sketch
in Figure~\ref{fig:system2} 
\begin{align*}
    E_\mathrm{new}-E_\mathrm{old}&=(-J_1L\!\cdot\!(2-2)-J_2L\!\cdot\!(-4))-(-J_1L\!\cdot\!4-J_2L\!\cdot\!4)\\
                                 &= -2J_1L-(-2J_1L)=0
\end{align*}
holds. For the right graphic one obtains
\begin{align*}
    E_\mathrm{new}-E_\mathrm{old}&=(-J_1L\!\cdot\!(3-1)-J_2L\!\cdot\!(2-2))-(-J_1L\!\cdot\!(3-1)-J_2L\!\cdot\!(2-2))\\
                                 &= -2J_1L-(-2J_1L)=0.
\end{align*}
Hence, the two processes do not change the energy of the system and one can construct with these two line flips a total of $2^L$ other
ground-states. For the vertical pattern one can construct in an analogous way $2^L$ ground-states, as well. Both cases contain the ferromagnetic 
ground-states, therefore the total number of ground-states generated has to be subtracted by $2$, finally leading to a degeneracy of 
$2^{L+1}\!-\!2$~\cite{Kalz2008}.


