

\section{Statistical physics treatment}
\label{sec:statistical_physics}

In this work the system under consideration is coupled with the environment through an exchange of energy, but not of particles.
Hence, the statistical equilibrium properties of the microstates can be described by the canonical ensemble, i.e., $\bm{\sigma}$ occurs at 
an environment temperature $T$ with a probability 
\begin{align} 
    P(\bm{\sigma})=Z_T^{-1}\exp\left(-\frac{\mathcal{H}(\bm{\sigma})}{kT}\right).
    \label{align:canonical_distribution}
\end{align}
where $Z_T\!\!=\!\!\sum_{\bm{\sigma}}\exp(-\mathcal{H}(\bm{\sigma})/kT)$ denotes the partition function and $k$ the Boltzmann constant. Therefore, 
the expectation value of an arbitrary observable $\mathcal{O}$ depends only on the temperature $T$ and its thermodynamic expectation value
can be computed by
\begin{align*}
    \langle \mathcal{O} \rangle = \sum_{\bm{\sigma}}\mathcal{O}(\bm{\sigma})P(\bm{\sigma}).
\end{align*}
The equilibrium treatment through the canonical ensemble requires the ergodicity, i.e., the expectation value $\langle \mathcal{O} \rangle_\text{time}$ 
obtained through the time evolution of the system coincides with $\langle \mathcal{O} \rangle_\text{phase space}$ received through the geometry of the 
phase space (Ergodic Theorem), in particular its energy surface. Or in more ingenuous words, the system reaches every microstate on the energy 
surface within a sufficiently long duration.~\cite{Landau1976,Schwabl2000} Therefore, Equation~\eqref{align:canonical_distribution} describes
only the thermodynamic equilibrium where the probabilities are independent of the trajectory in the phase space, which in general is not always the case.

Performing the summation over all possible microstates $\sum_{\bm{\sigma}}$ is generally difficult in practice. The density of states delivers a 
concept, to overcome this difficulty. For explanation, let's consider the possible eigenvalues of the Hamiltonian $E\!=\!\mathcal{H}(\bm{\sigma})$ and 
the possible eigenvalues of a microstate dependent observable $M\!=\!\mathcal{M}(\bm{\sigma})$. The density of states $\Omega(E,M)$ assigns 
the number of corresponding microstates to the values $E$ and $M$. The expectation value of observables depending on $E$ and $M$,
can then be rewritten as
\begin{align*}
    \langle \mathcal{O} \rangle = Z_T^{-1}\sum_{E,M}\mathcal{O}(E,M)\Omega(E,M)\exp\left(-\frac{E}{kT}\right).
\end{align*}
Thus, the observable implicitly depends on the microstates $\bm{\sigma}$. To avoid handling with a density of states depending on $E$ and $M$, a
dependency reduction can be performed through $\Omega(E)\!=\!\sum_{M}\Omega(E,M)$. To compute the above expectation value, an effective
energy dependent observable is necessary, received by
\begin{align}
    \langle \mathcal{O} \rangle &= Z_T^{-1}\sum_{E}\frac{\sum_M\mathcal{O}(E,M)\Omega(E,M)}{\sum_M\Omega(E,M)}\Omega(E)\exp\left(-\frac{E}{kT}\right) \nonumber \\
                                &= Z_T^{-1}\sum_{E}\mathcal{O}_\text{eff}(E)\Omega(E)\exp\left(-\frac{E}{kT}\right).
    \label{align:Lists}
\end{align}
The effective observable $\mathcal{O}_\text{eff}(E)$ is also known as list, an actual energy
dependency of $\mathcal{O}(E,M)$ is not required. Further, $\mathcal{O}_\text{eff}(E)$ is the expectation value of $\mathcal{O}(\bm{\sigma})$ with 
$\bm{\sigma}$ distributed according to the micro canonical ensemble (no $E$ exchange). This reduction procedure works with any observable $M$.~\cite{Janke2012}





\subsection*{Thermodynamic quantities}

To treat the studied thermodynamic quantities properly, consider the Hamiltonian~\eqref{align:Hamiltonian} with a non-vanishing external magnetic 
field $h$ aligned parallelly to the spins $\sigma_i$
\begin{align*}
    \mathcal{H}_h(\bm{\sigma})=-J_1\sum_{\langle ij \rangle}\sigma_i\sigma_j-J_2\sum_{[ij]}\sigma_i\sigma_j+h\sum_{i=1}^N\sigma_i.
\end{align*}
For simplicity, let $\mathcal{M}(\bm{\sigma})$ denotes the microstate dependent sum $\sum_{i=1}^N\sigma_i$. Using the modified Hamiltonian, define 
the thermodynamic potential
\begin{align}
    F(T,h)=-kT\ln\left(Z_{T,h}\right)=-kT\ln\left(\sum_{\bm{\sigma}}\exp(-\mathcal{H}_h(\bm{\sigma})/kT)\right),
    \label{align:FreeEnergy}
\end{align}
also known as canonical free energy. Its partial derivatives are given by
\begin{align}
    \partial_TF(T,h) &= -k\ln\left(Z_{T,h}\right)-\frac{kT}{Z_{T,h}} \sum_{\bm{\sigma}}\frac{\mathcal{H}_h(\bm{\sigma})}{kT^2}\exp\left(-\frac{\mathcal{H}_h(\bm{\sigma})}{kT}\right) \nonumber \\
                     &= -k\ln\left(Z_{T,h}\right)-\frac{\left\langle E \right\rangle}{T} \label{align:ener} \\
    \partial_hF(T,h) &= \frac{kT}{Z_{T,h}}\sum_{\bm{\sigma}}\frac{\partial_h\left(\mathcal{H}(\bm{\sigma})+h\mathcal{M}(\bm{\sigma})\right)}{kT}\exp\left(-\frac{\mathcal{H}_h(\bm{\sigma})}{kT}\right) \nonumber \\
                     &= -\left\langle M \right\rangle, \label{align:magn}
\end{align}
where the non-respected macroscopic state variables are kept fixed. Especially, both the internal energy $\left\langle E \right\rangle$ 
and the conventional magnetization\footnote{The term conventional is used, because some modified magnetizations are defined in the following.} 
$\left\langle M \right\rangle$ can be obtained through partial derivatives of the free energy $F$. 
Furthermore, the internal energy is a function received via a Legendre-transformation~\cite[p.79]{Schwabl2000} of the free energy, i.e., 
$\left\langle E \right\rangle\!(\partial_TF,h)=T\partial_TF(T,h)\!-\!F(T,h)$. Other relevant thermodynamic quantities are the specific heat $c_V$ and
susceptibility $\chi$ defined as the second partial derivatives of the free energy 
\begin{align}
    c_V  \!&=\! -\frac{T\partial_T^2F(T,h)}{N} 
          = \frac{T}{N}\left( - \frac{k}{Z_{T,h}}\!\sum_{\bm{\sigma}}\frac{\mathcal{H}_h(\bm{\sigma})}{kT^2}\exp\left(-\frac{\mathcal{H}_h(\bm{\sigma})}{kT}\right) 
                              + \frac{\langle E \rangle}{T^2} + \frac{\partial_T\langle E \rangle}{T}\right) \nonumber \\
         &= \frac{\sum_{\bm{\sigma}}\mathcal{H}_h(\bm{\sigma})\partial_T\exp\left(-\mathcal{H}_h(\bm{\sigma})/kT\right)}
                 {NZ_{T,h}}
           -\frac{\langle E \rangle\sum_{\bm{\sigma}}\partial_T\exp\left(-\mathcal{H}_h(\bm{\sigma})/kT\right)}
                 {NZ_{T,h}} \nonumber \\
         &= \frac{\langle E^2 \rangle-\langle E \rangle^2}{NkT^2}  \label{align:heat} \\
    \chi &= -\frac{\partial_h^2F(T,h)}{N} = \frac{\partial_h\langle M \rangle}{N} \nonumber \\
         &= \frac{\sum_{\bm{\sigma}}\mathcal{M}(\bm{\sigma})\partial_h\exp\left(-\mathcal{H}_h(\bm{\sigma})/kT\right)}
                 {NZ_{T,h}}
           -\frac{\langle M \rangle\sum_{\bm{\sigma}}\partial_h\exp\left(-\mathcal{H}_h(\bm{\sigma})/kT\right)}
                 {NZ_{T,h}} \nonumber \\
         &= \frac{\langle M^2 \rangle-\langle M \rangle^2}{NkT}. \label{align:susc} 
\end{align}
Therefore, the internal energy, conventional magnetization, specific heat and susceptibility can be expressed both in terms of partial derivatives of a 
thermodynamic potential  and in terms of expectation values. This fact justifies the importance of these observables for critical phenomena studies with 
computer simulations.~\cite{Schwabl2000}

The quantities above are these for a non-vanishing external field. Taking the limit $\lim_{h\to0}\mathcal{O}$ corresponding to 
$\mathcal{H}_h,Z_{T,h}\to\mathcal{H},Z_T$, the quantities are obtained for $h\!=\!0$.  





\section{Phase transition}
\label{sec:phase_transitions}

In thermodynamics there exists critical phenomena so-called phase transitions, where the system is subjected to significant property changes. Such 
transitions occur through changes in the symmetry of the system, i.e., changes to a state of more (or less) symmetry or changes between different unrelated 
symmetries~\cite{Landau1976}. The different states are called phases, the state with the lower symmetry is named ordered phase, as opposed to the 
disordered phase of higher symmetry. The engine of such transitions are variations of the macroscopic state variables. In the considered case 
these are the temperature $T$ and the external magnetic field $h$. To describe these phenomena quantitatively, an order parameter is commonly used, 
assigning to the ordered phase a finite value and vanishing for the disordered one. More generally, the order parameter is finite at one of the phases 
and vanishes at the other one, but such an order parameter does not always exist.~\cite{Landau1976,Schwabl2000}

For the system under consideration, a quantity of high interest is the two point spin correlation function
\begin{align*}
    G(i,t_1,j,t_2)=\langle \sigma_i(t_1)\sigma_j(t_2) \rangle - \langle \sigma_i(t_1) \rangle\langle \sigma_j(t_2) \rangle,
\end{align*}
where $t_1,t_2$ denotes two times in system history. To achieve a better understanding of this function, let's consider the term 
$1\!+\!\langle \sigma_i(t_1)\sigma_j(t_2) \rangle$, which is proportional to the probability of observing the $i^\text{th}$ spin at $t_1$
aligned equally to the $j^\text{th}$ spin at $t_2$~\cite{Stanley1987}. Hence, this correlation function evaluates the microscopic relations of the 
system so $G(i,t_1,j,t_2)$ quantifies the correlations between $\sigma_i(t_1)$ and $\sigma_j(t_2)$. The considered system has translation 
invariance, therefore $G$ is actually a function of the delay $j\!-\!i$. Two cases can be examined, firstly, at $t_1\!=\!t_2$, the two point spin correlation
function decays exponentially for large delays $j\!-\!i\gg 1$ according to
\begin{align*}
    G(j\!-\!i)\propto |a\cdot(j\!-\!i)|^{-\kappa}\exp\left(-\frac{a\cdot(j\!-\!i)}{\xi}\right),
\end{align*}
with the lattice constant $a$, a dimension and system dependent exponent $\kappa$ and the spatial correlation length $\xi$ modelling the effective 
affecting range of a spin~\cite{Janke2012}. Secondly, for $j\!-\!i=0$, it is convenient to average the spin values over 
the whole lattice, because no lattice site is preferable under periodic boundary condition. Furthermore, to normalize the resulting expression to unity 
for $t_1\!=\!t_2$ and afterward to take the thermodynamic expectation value. Thus, the general expression of the two point spin correlation function becomes
\begin{align}
    G(t_1,t_2) &= \left\langle \frac{\frac{1}{N}\sum_{i=1}^N\sigma_i(t_1)\sigma_i(t_2)-\frac{1}{N}\sum_{i=1}^N\sigma_i(t_1)\cdot\frac{1}{N}\sum_{i=1}^N\sigma_i(t_2)}
                                    {1-\frac{1}{N}\sum_{i=1}^N\sigma_i(t_1)\cdot\frac{1}{N}\sum_{i=1}^N\sigma_i(t_2)}\right\rangle \nonumber \\
               &= \left\langle \frac{N\sum_{i=1}^N\sigma_i(t_1)\sigma_i(t_2)-\mathcal{M}(\bm{\sigma}(t_1))\mathcal{M}(\bm{\sigma}(t_2))}
                                    {N^2-\mathcal{M}(\bm{\sigma}(t_1))\mathcal{M}(\bm{\sigma}(t_2))}\right\rangle.
    \label{align:spin_correlation}
\end{align}

In a mathematical sense, the free energy $F(T,h)$ is a non-analytical function at the phase transition. Hence, phase transitions can be characterized 
according to the discontinuities of the partial derivatives of $F(T,h)$. A phase transition is called discontinuous if $F(T,h)$ is continuous and at least 
one  of the first derivatives has a discontinuity. A continuous phase transition is classified through a continuous first partial derivatives and singularities
or discontinuities in the higher order.~\cite{Schwabl2000} Of course there are some exotic cases, which cannot be classified in a conventional fashion. It 
should be emphasized that these discontinuities and singularities actually appear for infinite system sizes only in the thermodynamic limit. For finite systems 
the free energy is usually analytical. Only finite system sizes are possible in computer simulation studies. Therefore, the results have to be extrapolated 
to $N\!\to\!\infty$.~\cite{Janke2012}

In the following, the conventional kinds of phase transitions are shortly described, as well as a representative of the exotic ones. Additionally, the discussions
are limited to the temperature driven phase transitions. Thus, the phase transition point is located through a critical temperature $T_c$.





\subsection*{Discontinuous phase transition}

As mentioned, this kind of phase transitions are characterized by discontinuous changes of the first partial derivative of the free energy. Hence, discontinuities
in the entropy $S\!=\!-\partial_TF(T,h)$ or magnetization $\langle M\rangle\!=\!-\partial_hF(T,h)$ suggest $\delta$-function singularities of the specific heat
or the susceptibility. The free energy is reflected by two different analytical functions
$F_1(T,h)$ for $T\!<\!T_c$ and $F_2(T,h)$ for $T\!>\!T_c$, satisfying the continuity condition $F_1(T_c,h)\!=\!F_2(T_c,h)$ at the transition point. 
Each function corresponds to a phase~\cite{Landau1976}. Thus, the 
transition point exhibits a coexisting state of both phases, implying a finite spatial correlation length $\xi$, i.e., there are finite regions 
corresponding to the phases  involved. In simulation studies, such phenomena are noticeable by double-peaked energy or magnetization histograms,
where each peak corresponds to one of the phases.~\cite{Janke2003}

The considered system is subjected to a discontinuous phase transition for $J_2/J_1\!\in\!(-g^*,-1/2)$ with $g^*\!\approx\!0.67$~\cite{Jin2012}, which is 
still under debate. This
critical behavior is supported by the work of Kalz et al.~\cite{Kalz2008} as well. Especially, they obtain the double-peaked histograms and argue
for a $\delta$-function singularity of the specific heat in the thermodynamic limit. For this case, the ground-states are super-antiferromagnetic as pointed 
out in Section~\ref{sec:GroundState}. This justifies the following choice of the order parameter
\begin{align}
    \mathcal{M}_s(\bm{\sigma}) = \begin{cases}
                                    \mathcal{M}_h(\bm{\sigma})=\sum_{i=1}^N(-1)^{x_i}\sigma_i & \text{for } |\mathcal{M}_h|>|\mathcal{M}_v| \\
                                    \mathcal{M}_v(\bm{\sigma})=\sum_{i=1}^N(-1)^{y_i}\sigma_i & \text{for } |\mathcal{M}_h|\le|\mathcal{M}_v|
                                 \end{cases}
    \label{align:stripped_magnetization}
\end{align}
where $x_i,y_i\!\in\!\{1,...,L\}$ index the columns and rows of the lattice. $\mathcal{M}_s$ assigns the super-antiferromagnetic ground-states with 
$\pm N$ and effectively vanishes for spin configurations distributed uniformly (corresponding to the high temperature phase). Furthermore, 
$\langle M_{s,h,v}\rangle$ can be considered as modified magnetizations and all previous expressions of quantities containing the conventional magnetization 
can be adjusted by replacing $\mathcal{M}$ with $\mathcal{M}_{s,h,v}$, like Equations~\eqref{align:susc} and~\eqref{align:spin_correlation}. 
In the following, the index $s,h,v$ is related to this very replacement. 





\subsection*{Continuous  phase transition}

The phase transitions of continuous kind have at least continuous changes of the free energy and their first partial derivatives. The ones being discontinuous
in their second derivatives in form of singularities have a major status and are being examined in the following. The corresponding free energy can be expressed 
by a single continuously differentiable function describing both phases. Therefore, the transition point exhibits a continuous mixture containing the symmetries 
of both phases, i.e., there is no coexisting of finite regions corresponding to the different phases and rather a single "transition phase" subjected to the whole 
system.~\cite{Landau1976,Schwabl2000} Hence, the spatial correlation length diverges $\xi\!\to\!\infty$ at $T_c$, according to the power law
$$ \xi \propto\left|1-T/T_c\right|^{-\nu}+\Delta_\xi, $$
where $\nu$ denotes a characteristic exponent also known as the critical exponent and $\Delta_\xi$ is a correction term. The 
singularities of the second partial derivatives of $F(T,h)$ and the order parameter (for the magnetic systems, $\langle M \rangle$) behaves according to 
power laws in the vicinity of $T_c$ as well,
$$ c \propto \left|1-T/T_c\right|^{-\alpha}+\Delta_c,$$
$$ \langle M \rangle \propto \left(1-T/T_c\right)^{\beta}+\Delta_M \  \text{ for } \  T<T_c, $$
$$ \chi \propto \left|1-T/T_c\right|^{-\gamma}+\Delta_\chi, $$
with the critical exponents $\alpha,\beta,\gamma$ and the correction terms $\Delta_{(\cdot)}$.~\cite{Janke2012} The possible values of the exponents 
$\alpha,\beta,\gamma,\nu$ define the universality of critical phenomena, i.e., in proximity of $T_c$, the microscopic details of the interactions becomes 
irrelevant and only universal properties, like dimension, symmetry of the order parameter or general character of the interaction are essential. 
Therefore, different systems with the same universal properties behave equally in the vicinity of $T_c$. These facts are supported by the renormalization
group theory.~\cite{Schwabl2000} In computer simulation studies, it is impossible to observe actual divergences, because only finite system sizes are 
possible such that the singularities appear rather as dislocated and rounded peaks, and it is necessary to extrapolate the finite size observations
to an infinite system size (thermodynamic limit). This can be done with the finite-size scaling approach $\xi\to L$, where $L$ is the linear system size 
(lattice length), implying the relation
\begin{align}
    \left|1-T/T_c\right| \propto L^{-1/\nu}.
    \label{align:FSS_power}
\end{align}
For finite systems the spatial correlation length is limited by $L$, justifying the replacement of $\xi$ with large $L$.~\cite{Janke2012} 

The discussion above holds for a huge number of systems but in some cases the phase transition exhibits a continuous character and a different spatial
correlation length divergence, i.e., the power law divergence of $\xi$ is not strong enough, causing an exponential divergence
\begin{align*}
    \xi \propto \exp\left(c\left| 1-T/T_c \right|^{-\overline{\nu}}\right) + \Delta_\xi,
\end{align*}
with some system dependent constant $c$ and a new exponent $\overline{\nu}$. Similarly, the finite-size approach can be followed according to the replacement 
of $\xi$ with large $L/L_0$, leading to the relation
\begin{align}
    \left|1-T/T_c\right| \propto \left( \ln\left(L/L_0\right)\right)^{-1/\overline{\nu}},
    \label{align:FSS_exponential}
\end{align}
with a system dependent characteristic length scale $L_0$. This idea is originated by a work of Kosterlitz~\cite{Kosterlitz1974}, studying the 
two-dimensional XY-model ($\overline{\nu}=1/2$).

Back to the considered system of this thesis, different papers~\cite{Kalz2008,Landau1980,Li2021} point out that the kind of the phase transition is 
continuous for the system with $J_2/J_1\!\in\!(-1/2,0]$. The corresponding order parameter is the conventional magnetization $\mathcal{M}(\bm{\sigma})$. 
Furthermore, they found the same set of critical exponents as the conventional Ising model ($J_2/J_1=0$), suggesting
a membership of the Ising universality class. Hence, all mentioned variations of the considered system behave equally in proximity of there 
respective transition point. It should be emphasized that the critical exponents are equal but the transition temperatures $T_c$ differ. 





\subsection*{Glassy phase transition}

These kinds of phase transition are subjects of recent researches. There are dynamical and static approaches to treat these phase transitions in the context 
of statistical physics. In the low temperature phase, the difficulty of this kind originates from the non-ergodic behavior and the huge number of deep free 
energy valleys (metastable states), i.e., the system is easily trapped in a sub-state space and effectively freezes in this state. 
Especially, the states minimizing the free energy in the state space have a random character, thus the corresponding microstates have no significant 
similarities and appear disordered or asymmetrical (no ordered patterns in a conventional sense such as the ferromagnetic or super-antiferromagnetic ones). 
Usually, the freezing into a metastable state results from competing interactions. Conventionally, the term 
"glassy" is distinguished in spin glasses and structural glasses.~\cite{Binder1986,Mezard2000}

A spin glass is commonly characterized by freezing into a metastable state and quenched disorder, i.e., the system depends 
explicitly on random variables that are independent of time on all experimental
time scales~\cite{Binder1986}. In context of spin glasses the freezing property is also known as
frustration. The latter property can be explained by the Edwards-Anderson model defined as the Hamiltonian
\begin{align*}
    \mathcal{H}(\bm{\sigma})=-\sum_{\langle ij \rangle}J_{ij}\sigma_i\sigma_j,
\end{align*}
with Gaussian-distributed $J_{ij}$. The thermodynamic quantities of interest are obtained through their averages over different system realizations 
corresponding to a set of $J_{ij}$ drawn randomly, also known as replica. Hence, the disordered appearance of the metastable states or "random-valleys"
in the free energy results from explicit random variables in the Hamiltonian.~\cite{Binder1986} However, some newer publications, like 
Ref.~\cite{Westfahl2001,Kamber2020}, argue for spin glass without quenched disorder, i.e., self generated randomness.



A structural glass effectively freezes into an unsymmetrical microstate as well, denoted in this context as an amorphous state. However, their origin is 
different, the randomness or disorder of the metastable states is self generated like the arrangement of the atomic bonds in chemical glass. Therefore, the 
"random-valleys" characteristic of the free energy is a system related property and less artificial than in spin glasses, i.e., the main difference
to spin glasses is the absence of quenched disorder. Thus, the replicas result from independent system preparations and not from a new set of random 
variables.~\cite{Mezard2000}

Kirkpatrick et al. begin to blur the boundaries between spin glasses and structural glasses in their papers~\cite{Kirkpatrick1986,Kirkpatrick1987},
mentioning analogies between structural glasses and Potts spin glasses. They list similarities to discontinuous phase transitions and 
suggest possible glassy behavior of frustrated regular spin problems. Furthermore, these kinds of
problems are named discontinuous spin glasses, which exhibits a continuous change of the entropy, internal energy and a discontinuous
change of the order parameter~\cite{Mezard2000}. A general theory containing these kinds of problems is developed in a paper of Westfahl et al.~\cite{Westfahl2001}. 

This analytical approach predicts a possible temperature range of the glass transition temperature $T_g$, where the system effectively freezes into a 
metastable state, i.e., $T_g\!\in\!(T_K,T_A)$ with the limit temperatures $T_K,T_A$ of glassy behavior. M\'ezard et al.~\cite{Mezard2000} mention a relaxation 
time $\tau$ divergence of the spin autocorrelations 
at $T_g$ where the explicit value of this temperature depends on the cooling rate~\cite{Westfahl2001}. The lower boundary $T_K$ corresponds to the ideal 
glass transition temperature in the infinite cooling rate limit~\cite{Timmons2018}. A connection between $\tau$ and the limit temperature $T_K$ is
given by the Vogel-Fulcher law~\cite{Kirkpatrick1987,Westfahl2001}
\begin{align}
    \tau\propto\exp\left(\frac{c\cdot T_K}{T-T_K}\right).
    \label{align:Vogel_Fulcher}
\end{align}
The upper boundary $T_A$ exhibits the occurrence temperature of the metastable state~\cite{Kirkpatrick1987}, i.e., 
these states appear with a non-negligible probability, resulting in significantly increasing spin autocorrelations, because the system is subjected to 
strong memory effects.

The nature of the system under consideration for $J_2/J_1\!=-\!1/2$ is unclear in the low temperature range (nonzero $T$). There are some evidence supporting
a glassy behavior depicted above~\cite{Timmons2018}. Firstly, the huge number of metastable states (degeneracy), Section~\ref{sec:GroundState} and the frustration 
arising from the competition between the  NN-term and repulsive NNN-term in the Hamiltonian~\eqref{align:Hamiltonian}. Secondly, discontinuous spin glasses 
have both a continuous and discontinuous character, which may connect the two regimes for $J_2/J_1\!\in\!(-g^*,-1/2)$ and $J_2/J_1\!\in\!(-1/2,0]$.
However, there are also evidences against it, like the absence of the random or disordered characteristic of the metastable states, and 
especially the ordered form of the ground-states~\cite{Kalz2008,Lee2024}.










