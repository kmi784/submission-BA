
\section{Data interpretation}

\label{sec:discussion}

The procedure of data collection is separated from the interpretation, because the collected data is indisputable, but the interpretation is up for debate
and does not deliver a final conclusion. In this Section~\ref{sec:discussion}, all simulation results of the previous two Sections are analyzed and connected to the physical 
properties depicted in Chapter~\ref{cha:physics}, particularly to the statistical physics treatment and to the
studies of critical phenomena, mentioned in Section~\ref{sec:statistical_physics} and~\ref{sec:phase_transitions}.

\subsection*{Equilibrium properties}

Considering the energy dependent observables in Figure~\ref{fig:energy}, the behavior of the internal energy per lattice site as a function of $T$ exhibits 
an inflection point with a positive slope. This implies a maximum of its first derivative, more precisely of the specific heat $c_V$ according to the 
first equal sign in Equation~\eqref{align:heat}. Generally, a maximum in the specific heat data for finite system sizes may indicate critical phenomena,
which is not always the case, e.g., the one dimensional Ising model. Especially the narrowing and the slightly growing when increasing $L$ suggests a 
divergence in the thermodynamic limit. The movement of the maximum locations proposes $T_\mathrm{max}\!=\!0$ for $N\!\to\!\infty$, i.e., the system does not 
undergo a phase transition at a finite temperature. To confirm this presumption, the data in Table~\ref{table:heat} is fitted to two different laws 
connecting the temperature near criticality with the spatial correlation length, i.e., the lattice length $L$ for the finite systems under consideration.

The first law is originated by a paper of Kalz et al.~\cite{Kalz2008}, suggesting a power divergence of the spatial correlation length according to the 
continuous phase transitions with singularities in the second derivatives of the free energy. With Equation~\eqref{align:FSS_power} under the assumption 
of $T_c\!=\!0$ and the redefinition $\rho\!=\!-1/\nu$, the corresponding fit function results in
\begin{align*}
  T_\mathrm{max}(L)=A\cdot L^{\rho},
\end{align*}
where $A$ denotes some proportional constant. 

\begin{figure}[!h]
  \includegraphics{fits_Tmax.eps}
  \caption{The mentioned fit approaches are plotted in this figure with rescaled axes such that the corresponding law behaves linear.}
  \label{fig:fits_heat}
\end{figure}

The second law is introduced by a work of Lee et al.~\cite{Lee2024}, predicting
an exponential divergence in the fashion of Equation~\eqref{align:FSS_exponential}. Landau et al. study the considered system for 
varying $J_2/J_1$ ~\cite{Landau1980}. They conclude $\overline{\nu}\!\approx\!1$ in an analysis of the critical temperature as function of these ratios 
$T_c(J_2/J_1)$. With that and the assumption of $T_c\!=\!0$, the second fit function becomes 
\begin{align*}
  T_\mathrm{max}(L)=A\cdot\ln(L/L_0)^{-1},
\end{align*}
with some proportional constant $A$ and the characteristic length scale $L_0$. 

The corresponding fits are realized with the least square approximation described in Section~\ref{sec:LeasatSquare}. 
The best fits could be observed by taking finite size effects into account, i.e., the smallest lattice length $L\!=\!30$ is neglected. They are 
shown in Figure~\ref{fig:fits_heat}, the associated model parameters and $\chi^2$-values are depicted in Table~\ref{table:fits_heat}.


The model of the exponential divergence shows a normalized $\chi^2$-value close to the ideal value $1$, suggesting a model 
representing the observed data. Further, the corresponding model parameters agree with the depicted one in Ref.~\cite{Lee2024}, 
$L_{0,\mathrm{Lee}}\!=\!e^{-0.736}\!=\!0.479$. 


The normalized $\chi^2$-value of the power divergence approach is much larger than the ideal one, implying 
that the model does not represent the observed data. Unfortunately, its determined model parameters deviates from the presented ones in Ref.~\cite{Kalz2008}, 
$A_\mathrm{Kalz}\!=\!0.881(1)$ and $\rho_\mathrm{Kalz}\!=\!-\!0.214(1)$, even taking all of the uncertainties into account. The best agreements to the 
presented ones in Ref.~\cite{Kalz2008} could be observed by neglecting finite size effects\footnote{Taking $L\!=\!30$ into account and accepting higher $\chi^2$-values 
results in $A\!=\!0.878(15)$, $\rho\!=\!-0.2123(40)$ and $\chi^2/\mathrm{dof}\!=\!68.137$ for the power divergence.}.
\\

\begin{table}[h]
  \centering
    \begin{tabular}{|lll|}
      \multicolumn{3}{l}{power divergence} \\
      \hline
      $A$ & $\rho$ & $\chi^2/\mathrm{dof}$ \\
      \hline
      $0.838(11)$ & $-0.2020(31)$ & $19.042$ \\
      \hline
      \multicolumn{3}{l}{}\\
      \multicolumn{3}{l}{exponential divergence} \\
      \hline
      $A$ & $L_0$ & $\chi^2/\mathrm{dof}$ \\
      \hline
      $1.7609(87)$ & $0.485(12)$ & $2.08$ \\
      \hline
    \end{tabular}
  \caption{The model parameters and $\chi^2$-values of the least square approximations are shown for the corresponding divergence law.}
  \label{table:fits_heat}
\end{table}


The final consideration treats the magnetic observables for the ferromagnetic and super-antiferromagnetic cases in Figure~\ref{fig:magnet}. 
The associated magnetization per lattice site increase slightly when lowering temperature until a plateau is reached.
Perhaps, this behavior results from coexisting domains of ferromagnetic and super-antiferromagnetic ordering similarities. Further, the numerical
values of the plateaus decrease when increasing $L$. 

This can be explained by the consideration of the observables $|\mathcal{M}_{(\cdot)}(\bm{\sigma})|$ 
as the square root of the random variable $Y_{(\cdot)}$ defined by
$$Y_{(\cdot)}\!=\!\sum_{i=1}^{N}\sigma_{(\cdot),i}^2,$$
where $\sigma_{(\cdot),i}$ denotes the conventional or super-antiferromagnetic spin values. According to the Central Limit Theorem~\cite[p.246]{Behrends2013}, it should 
$\langle Y_{(\cdot)}\rangle\!=\!N\langle\sigma_{(\cdot),i}^2\rangle$ holds with the expectation value $\langle\sigma_{(\cdot),i}^2\rangle$, being independent of $N$.
Its square root can be estimated by $(\hat{Y}_i/N)^{1/2}\!=\!M_{(\cdot)}/\sqrt{N}$. Hence, the plateaus of the magnetization divided by $\sqrt{N}$ coincides at a 
numerical value, shown in Figure~\ref{fig:magnet_disscussion}. 

The corresponding susceptibilities as function of $T$ show a declining exceed but no peak 
in the temperature range with significant behavior of $E$ and $c$.

\begin{figure}[!h]
  \includegraphics{magnet_disscussion.eps}
  \caption{This figure depicts the presumed coinciding of the magnetization plateaus.}
  \label{fig:magnet_disscussion}
\end{figure}



\subsection*{Dynamical properties}

The measured spin autocorrelation function $G_{(\cdot)}(t_1,t_2)$, Equation~\eqref{align:spin_correlation}, evaluates changes of 
similarities after a duration under the premise of ordering, i.e., $G_{(\cdot)}(t_1,t_2)$ becomes $1$ if 
$\bm{\sigma}(t_1)$ and $\bm{\sigma}(t_2)$ appear equally similar or equally dissimilar under the corresponding premise 
and becomes less $1$ if these states have changes in their similarities. The above ordering premise is associated to the ferromagnetic, 
horizontal and vertical striped patterns. In the following, the data of $G_{(\cdot)}(t_1,t_2)$, received by the non-equilibrium protocols described in 
Section~\ref{sec:timmons_experiment}, is interpreted for the warming run (high $T$ quench) and cooling run (low $T$ quench). During these
examinations, the procedure is performed for single spin update, to check the consistency to the findings in Ref.~\cite{Timmons2018}. Further, this routine is 
repeated for the line spin update, to conclude that the findings in Ref.~\cite{Timmons2018} are perhaps indications of a glassy behavior or of a 
dynamic effects of the used simulation algorithm.

\subsubsection*{High temperature quench}

In the following, the term "exceeding range" intends to the temperature range where $G_{(\cdot)}(t_1,t_2)$ 
increases effectively from $0.1$ to $0.9$ and the explanation refer to the spin autocorrelations $G_{(\cdot)}(0,t^*)$ of the single spin update in 
Figure~\ref{fig:auto} (left column). The exceeding range moves slowly to lower $T$ when increasing the time $t^*$ of the measurement.

For small 
$t^*$ in the exceeding range, there are system size dependent differences of $G_{(\cdot)}(0,t^*)$ noticeable. A system with a smaller lattice 
length $L$ appears more correlated for higher temperatures $T$, which may be originated by a smaller absolute number of spin flip proposals 
(sweeps $\propto L^2$) during a nearly unvarying acceptance rate, Figure~\ref{fig:prop}.

This behavior should be fading for a higher statistics, 
and indeed it is the case for larger $t^*$. All of the $G_{(\cdot)}(0,t^*)$ remain in excess for low $T$ at large $t^*$, originated by the 
frustration of the system.

Considering the plot of $G_{(\cdot)}(0,t^*\!=\!90000)$ in Figure~\ref{fig:auto} (left column), the ferromagnetic spin 
autocorrelation drops slightly at lower $T$ for the warming run with a ferromagnetic initialization. Therefore, the pattern at $t^*\!=\!90000$ slightly 
deviates in average from the initial one, resulting in a small decline of $G_{(\cdot)}(0,t^*)$. In comparison, the $G_{(\cdot)}(0,t^*\!=\!90000)$ 
of the other ordering premises (striped) remains at a value close to $1$. Hence, the patterns of the microstates after $0$ and $90000$ sweeps appear equally 
dissimilar under the striped ordering premises, because for a decrease of the striped spin autocorrelations, the system has to spontaneously order 
(in average) in a striped state, being highly unlikely for systems with a high degeneracy $2^{L+1}\!-\!1$, Section~\ref{sec:GroundState}. Or simply put, 
it is more likely for the system to transits into one of the many dissimilar states, then into one of the few similar states. It should be emphasized that 
an analogous behavior could be observed for horizontal or vertical striped initialization. 
\\
\begin{figure}[!h]
  \includegraphics{autoc_vs_accep.eps}
  \caption{This plot combines the spin autocorrelations at $t^*\!=\!900$ and acceptance rates as a function of the temperature for the warming run of the 
           line spin update. The right vertical axis refers to the values of the spin correlations (squares) and the left one to the logarithm of the 
           acceptance rates (lines).}
  \label{fig:autoc_vs_accep}
\end{figure}

Considering the spin autocorrelations of the line spin update in Figure~\ref{fig:auto} (right column), there are significant differences for small temperatures 
to the single spin update visible. For low $T$, all of the spin autocorrelations $G_{(\cdot)}(0,t^*)$  drop immediately to $0$, because the line flip is accepted with a high 
probability, Figure~\ref{fig:prop}, and they easily transit the system between the metastable states, Section~\ref{sec:GroundState}. This drop is originated 
by the fact that a short sequence of line flips generates easily all the considered ordering patterns, 
Section~\ref{sec:GroundState}. Hence, their application  destroys the similarities under the considered ordering premises within a few sweeps, resulting in
a decrease of $G_{(\cdot)}(0,t^*)$. 

The spin autocorrelations are only nonvanishing for small $t^*$, showing peaks. The following explanation connects this 
behavior to the used update algorithm, referring to Figure~\ref{fig:autoc_vs_accep}.

For the transitions between high temperature states (high $T$), a smaller amount of line flips in the update (spin and line flips) sequence results in smaller free energy 
barriers. This fact and the relatively high acceptance rate of the spin flips  reduces the numerical value of $G_{(\cdot)}(0,t^*)$. 

When lowering the temperature, the frustration emerges markedly and increases the spin autocorrelations until the line flip acceptance rate exceeds the 
spin flip acceptance rate. A subsequent decline of $G_{(\cdot)}(0,t^*)$ to $0$ is the result of the further growing line flip acceptance rate. 

When 
increasing $L$, the peak enlarges and moves to lower $T$, caused by the movement 
of the intersection point of the corresponding acceptance rates. This movement is originated by lower energy costs of line flips for smaller $L$, 
effecting a higher acceptance probability with fixed $T$ ($\propto\exp(-\Delta E/kT)$), Section~\ref{sec:MCsimulations}. Thus, for small $L$ the 
line flip acceptance exceeds the spin flip acceptance at higher $T$. 

Another, finding is the slight increase of the $G_{(\cdot)}(0,t^*)$ errors, 
when the line flip update markedly emerges. It should be emphasized that all of the observations are independent of the initialization state and of 
the considered ordering premise.


\subsubsection*{Low temperature quench}

Considering the obtained relaxation times for the dependent cooling run in Figure \ref{fig:relax}, the data of the single spin update shows a 
rapid increase when lowering $T$. 

This suggests a divergence, and therefore an indication of a glassy behavior, Section \ref{sec:phase_transitions}.
In analogy to Ref.~\cite{Timmons2018}, the nature of the divergence is assumed according to a power law, leading to the fit function 
\begin{align*}
  \tau(T)=A|T-T_g|^\rho.
\end{align*}
Performing the least square approximation with this model leads to an estimate of glass transition temperature $T_g$. The same data is fitted to 
the Vogel-Fulcher law~\eqref{align:Vogel_Fulcher} as well as in Ref.~\cite{Timmons2018}, implying the fit model
\begin{align*}
  \tau(T)=A\exp(c/(T-T_K)).
\end{align*}
This delivers a method to estimate the ideal glass transition temperature $T_K$, Section~\ref{sec:phase_transitions}. 

The corresponding fits and determined
parameters are depicted in Figure~\ref{fig:fits_relax} and Table~\ref{table:fits_relax}. All of the obtained model parameters are slightly out of the 
tolerances in comparison to the values in Ref.~\cite{Timmons2018}, $T_g\!=\!0.261(1)$, $\rho\!=\!3.38(2)$ and $T_K\!=\!0.091(5)$. Perhaps, this results from
the low statistics of the used data, because the means and the errors are obtained by only four different runs\footnote{The low statics is accepted to 
limit the corresponding computing time to 6 days.}. The deviations of $T_g$ may be caused by a differently chosen cooling rate then in Ref.~\cite{Timmons2018}.
\\ 

\begin{figure}[!h]
  \includegraphics{fits_relax.eps}
  \caption{The full logarithmic plots show the corresponding relaxation times as function of a rescaled temperature and the determined fit models.}
  \label{fig:fits_relax}
\end{figure}


\begin{table}[h]
  \centering
    \begin{tabular}{|llll|}
      \multicolumn{4}{l}{power law} \\
      \hline
      ordering premise & $A$ & $T_g$ & $\rho$ \\
      \hline
      ferromagnetic       & $3.67(72)$ & $0.2545(41)$ & $3.82(15)$ \\
      horizontal striped & $3.36(61)$ & $0.2530(39)$ & $3.90(15)$ \\
      vertical striped   & $3.32(60)$ & $0.2526(39)$ & $3.91(15)$ \\
      \hline
      \multicolumn{4}{l}{}\\
      \multicolumn{4}{l}{Vogel-Fulcher law} \\
      \hline
      ordering premise & $A$ & $T_K$ & $c$ \\
      \hline
      ferromagnetic       & $4.1(1.5)$ & $0.135(11)$ & $1.91(18)$ \\
      horizontal striped & $3.6(1.3)$ & $0.131(10)$ & $1.98(17)$ \\
      vertical striped   & $3.5(1.2)$ & $0.131(10)$ & $1.98(17)$ \\
      \hline
    \end{tabular}
  \caption{This table depicts the received parameters for the corresponding models.}
  \label{table:fits_relax}
\end{table}

In comparison to $\tau$ of the single spin update, the relaxation times of the line spin update show no continuing increase rather an increase 
with a subsequent sharp decline, Figure~\ref{fig:relax}. This behavior can be explained by considering Figure~\ref{fig:relax_vs_accep}, depicting 
additionally the acceptance rates to the relaxation times. 

It is noticeable that the relaxation times curves split when the line spin update 
acceptance exceeds the single spin update one by some magnitudes. Hence, the visible differences result from the emerging line spin update as well.


However, the measured values of the line spin update relaxation times remain at large numerical values, $\tau\!=\!4584(1358)$ at $T\!=\!0.31$, but 
show no singular behavior.
\\

\begin{figure}[!h]
  \includegraphics{relax_vs_accep.eps}
  \caption{This plot combines the relaxation times and acceptance rates as a function of the temperature for $L\!=\!128$. The right vertical axis refers to
  the values of the relaxation times (green) and the left one to the logarithm of the acceptance rates (black).}
  \label{fig:relax_vs_accep}
\end{figure}

In a visual way, this is understandable through the snapshots in Figure~\ref{fig:snap}. Both algorithms transit the 
system in near ground-state at low temperature within a sufficiently large number of sweeps. 

However, it is hard to sample ground-state patterns
with the single spin update, originated by the high energy costs of the necessary spin flip sequence resulting in an effectively freezing into an 
explicit state. 

On the other hand, the correct line flips cost no energy, and therefore the simulated system easily transits between ground-states, 
implying a good ground-state sampling. This point of view is supported by the behavior of the corresponding acceptance rates, Figure~\ref{fig:prop}.   

\begin{figure}[!h]
  \centering
  \includegraphics{snap.eps}
  \caption{This graphics shows snapshots for the single and line spin update at $T\!=\!0.3$ after $N_\mathrm{equi}\!=\!5\!\cdot\!10^{7}$ sweeps 
           for a system initialized in a high temperature state with lattice length $L\!=\!120$.}
  \label{fig:snap}
\end{figure}

\newpage




