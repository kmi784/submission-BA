\label{cha:results}


In this thesis, the Monte Carlo simulation data is obtained using MCMC methods depicted in Subsection \ref{sec:MCsimulations}. Particularly, 
the described single spin update and line spin update algorithms are used to simulate the system under consideration at different temperatures $T$. For the 
numerical work the units are chosen such that $J_1\!=\!k\!=\!1$. Further, $T$ is measured in units of $J_1/k$. In this Chapter \ref{cha:results}, the term $M_{(\cdot)}$ refers to the estimates of the observable $|\mathcal{M}_{(\cdot)}(\bm{\sigma})|$.





\section{Equilibrium properties}
\label{sec:kalz_experiment}

For the data collection of the free energy derivatives, the line spin update algorithm is used only. The system is simulated for the 
lattice lengths $L\!\in\!\{30,40,50,...,120\}$ and the NNN-interaction constant $J_2\!=\!-0.5$, where the simulation parameters are chosen according to the
Tables~\ref{table:experiment_kalz1} and~\ref{table:experiment_kalz2}. Additionally, some consistency checks are performed for $L\!\in\!\{50,80,100\}$ and 
$J_2\!\in\!\{-0.3,-0.6\}$, Appendix~\ref{sec:kalz_checks}. The corresponding data is received through a single independent cooling run, i.e., 
a different PRNG seed is used for each temperature. Furthermore, the system is initiated in a uniformly distributed state
and is cooled to the corresponding temperature within a sufficient number of equilibration sweeps.

\begin{figure}[!h]
  \includegraphics{energy.eps}
  \caption{The plots show the energy per lattice site (top) and specific heat (bottom) as function of the temperature for different lattice length.
           The squares refer to the means and errors and the lines to the WHAM data.}
  \label{fig:energy}
\end{figure}

Then, the internal energy $E$, Equation~\eqref{align:ener}, the conventional magnetization $M$, Equation~\eqref{align:magn}, specific heat $c_V$,
Equation~\eqref{align:heat} and susceptibility $\chi$, Equation~\eqref{align:susc} are measured and shown in Figures~\ref{fig:energy} and~\ref{fig:magnet}. 
Additionally, the data of the magnetization and susceptibility in the super-antiferromagnetic case Equation~\eqref{align:stripped_magnetization}
are depicted as well, 
Figure~\ref{fig:magnet}. 

\begin{figure}[!h]
  \includegraphics{magnet.eps}
  \caption{The magnetization per lattice site and susceptibility are shown in this figure, the left column refers to the 
           ferromagnetic observables and the right column to the super-antiferromagnetic ones.}
  \label{fig:magnet}
\end{figure}

In the plots, all the means and errorbars are obtained by the Jackknife analysis described in Subsection~\ref{sec:error}. In addition,
the density of states is estimated by WHAM, Section~\ref{sec:wham}, and hence the smooth curves of $E$ and $c$ could easily be received. For the observables 
depending on the magnetization, the lists defined by Equation~\eqref{align:Lists} are stored for each energy-bin in addition to the histograms. 
Thus, the resulting effective observables and the estimated density of states are combined to smooth curves of $M,M_S$ and $\chi,\chi_S$. 

Further, the peak locations of the specific heat data are analyzed more accurately. The numerical values of the maxima and their errors are shown 
in Table~\ref{table:heat}. They are obtained by measuring smaller histograms during the simulation run. Afterwards, the entire histogram of a simulation point 
is subtracted by these smaller histograms, resulting in a set of Jackknife histograms. These Jackknife histograms are combined to a set of densities
of states estimated by WHAM, succeeding in a set of specific heat curves and their peak locations $(T_\mathrm{max},c_{V,\mathrm{max}})$. The mean of
these maximums is determined by the conventional average and their error are computed taking the square root of Equation~\eqref{align:Jackknife}.

\begin{table}[h]
  \centering
    \begin{tabular}{|lllll|}
      \hline
      $L$ & $T_\mathrm{max}$ & $c_{V,\mathrm{max}}$ & $T_\mathrm{max}$ in Ref.~\cite{Kalz2008} & $T_\mathrm{max}$ in Ref.~\cite{Lee2024} \\
      \hline
      $30$  & $0.42946(24)$ & $0.8187(27)$  & $0.4289$  & $0.4293$ \\ 
      $40$  & $0.39969(27)$ & $0.8289(53)$  & $0.3988$  & $0.3996$ \\ 
      $50$  & $0.37933(31)$ & $0.8497(88)$  & $0.3798$  & $0.3799$ \\ 
      $60$  & $0.36492(38)$ & $0.874(13)$   & $0.3668$  & $0.3653$ \\ 
      $70$  & $0.35374(33)$ & $0.869(16)$   & $0.3536$  & $0.3541$ \\ 
      $80$  & $0.34522(27)$ & $0.884(16)$   & $0.3449$  & $0.3447$ \\ 
      $90$  & $0.33692(33)$ & $0.891(19)$   & $0.3356$  & $0.3373$ \\ 
      $100$ & $0.33054(25)$ & $0.899(18)$   & $0.3311$  & $0.3307$ \\ 
      $110$ & $0.32483(24)$ & $0.958(22)$   & $-$       & $0.3256$ \\ 
      $120$ & $0.31968(23)$ & $0.945(28)$   & $0.3188$  & $0.3209$ \\ 
      \hline
    \end{tabular}
  \caption{This table depicts the specific heat peak locations corresponding to the lattice lengths and the reference data.}
  \label{table:heat}
\end{table}



\section{Dynamical properties}
\label{sec:timmons_experiment}

For the studies of the spin autocorrelation function $G(t_1,t_2)$, described by Equation~\eqref{align:spin_correlation} and their vertical and 
horizontal striped modifications, both the single spin update and the line spin update are employed. In the following, 
$t_1\!=\!0$ refers to the time of the beginning data collection, i.e., $\bm{\sigma}(t_1\!=\!0)$ corresponds to the state of the system after the 
equilibration period. In this fashion, $t_2$ refers to the time of the measurement in units of single spin sweeps, i.e., only the single spin flip proposals 
contributes to a sweep. The system is simulated for the lattice length $L\!\in\!\{32,64,128\}$ and the NNN-interaction constant 
$J_2\!=\!-0.5$ by a dependent run in analogy to Ref.~\cite{Timmons2018}, i.e., the system is initialized in state 
referring to a low (or high) temperature, before it is slowly cooled (or warmed) to the nearest temperature and so on. A sufficient 
number of equilibration sweeps are performed between each temperature change, Figure~\ref{fig:protocol}. All of the data presented and in particular the errors are obtained by multiple
dependent runs with different PRNG seeds. Hence, the thermodynamic expectation value of the measured autocorrelation function Equation~\eqref{align:spin_correlation}
is substituted by the configurational expectation value. This replacement and method of error observation is motivated by the possibility that the 
system undergoes a glassy transition for $J_2\!=\!-0.5$. Therefore, the equilibrium assumption of the probability distribution underlying
the measurements  may be violated as well as the time translation invariance of the autocorrelation function mentioned in Section~\ref{sec:error}. In other
words $G(t_1,t_2)$ is not a function of the time delay $t_2\!-\!t_1$ and the expectation value appearing in $G(t_1,t_2)$ cannot be estimated by a single
run. 

\begin{figure}[!h]
  \includegraphics{protocol.eps}
  \caption{The figure depicts the scheme of data collection for the high $T$ quench (warming run) and the low $T$ quench (cooling run), 
           where $N_\mathrm{m}$ denotes the sweeps for data collection of $G(t_1,t_2)$ and $N_\mathrm{c}$ denotes the number of equilibration sweeps.}
  \label{fig:protocol}
\end{figure}

The dependent warming runs are performed with the ferromagnetic, horizontal and vertical striped initial states (Figure~\ref{fig:protocol}), the corresponding simulation parameters are 
chosen according to Table~\ref{table:experiment_timmons}. The values of the spin autocorrelation functions are measured for varying temperatures at $t_1\!=\!0$ 
and different $t_2$. The data shows no significant changes for different initial states, and thus only the case with the ferromagnetic initial state is shown in 
Figure~\ref{fig:auto}. Additionally, the acceptance rate $r_\mathrm{a}$ of the single spin flip proposals and the line flip proposals are measured and
shown in Figure~\ref{fig:prop}.
 
\begin{figure}[!h]
  \includegraphics{autocorrelation.eps}
  \caption{This plot shows the different variations of the spin autocorrelation functions at $t_1^*\!=\!900$, 
           $t_2^*\!=\!9000$ and $t_3^*\!=\!90000$ as a function of $T$ for the corresponding update algorithms.}
  \label{fig:auto}
\end{figure}

\begin{figure}[!h]
  \includegraphics{acceptance.eps}
  \caption{The acceptance rates are shown for the corresponding simulation algorithms and the lattice length.}
  \label{fig:prop}
\end{figure}
 
For the dependent cooling run, the simulated system size is $L\!=\!128$ with the depicted simulation parameters in Table~\ref{table:experiment_timmons}. The system under consideration
is initiated in a high temperature state, corresponding to spin values distributed uniformly. Afterwards, it is cooled to the considered temperatures and the 
corresponding spin autocorrelation functions are measured analogously to the warming run (Figure~\ref{fig:protocol}). Their relaxation times $\tau$ are defined as the time the autocorrelation 
function falls below the numerical value $0.01$ for a fixed temperature, in analogy to Timmons et al.~\cite{Timmons2018}. The autocorrelation functions are approximated 
according to the exponential decay, Relation~\eqref{align:autoTimeExp}, using ranged least square approximations, Section~\ref{sec:LeasatSquare}, with the 
model $f(t_2)\!=\!A\exp(-t_2/\tau_\mathrm{exp})$. Then, only the range of $G(0,t_2)$ exponentially behaving contributes to the fits, obtained by half logarithmic plots. 
Rearranging the resulting model, which is set to $0.01$ leads to 
\begin{align*}
    \tau          &= \tau_\mathrm{exp}\ln(100A) \\
    \sigma^2_\tau &= \ln(100A)^2\sigma_{\tau_\mathrm{exp}}^2+\left(\frac{\tau_\mathrm{exp}}{A}\right)^2\sigma_A^2
                                                            +\frac{\tau_\mathrm{exp}\ln(100A)}{A}(\sigma^2_{A,\tau_\mathrm{exp}}+\sigma^2_{\tau_\mathrm{exp},A}),
\end{align*}
with the model parameters $A$, $\tau_\mathrm{exp}$. The single indexed $\sigma^2$ denotes the variances and the double indexed ones the covariances. 
The variance of $\tau$ is obtained by the error propagation Formula~\eqref{align:Error_propagation}. The received relaxation times are shown in Figure~\ref{fig:relax}.
The approach is intended by avoiding storing large amounts of autocorrelation function values until the value $0.01$ is reached, which is nontrivial, especially 
for low temperatures. Usually, the number of necessary data is smaller for receiving reliable fits.

\begin{figure}[!h]
  \includegraphics{relax.eps}
  \caption{The relaxation times of the corresponding ordering premise as function of the temperature is shown for $L\!=\!128$ and the used simulation
           algorithms.}
  \label{fig:relax}
\end{figure}

\newpage




