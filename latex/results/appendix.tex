
\chapter{Derivatives of the free energy}
\label{app:kalz}

\section{Data generation details}

This section gives a compiling and execution instruction to reproduce the numerical work of the experiment in Section \ref{sec:kalz_experiment}. 
Firstly, the submission folder
\verb|submission/| contains four subfolders. The necessary codes are in \verb|generateK/| and \verb|analyzeK/| for the mentioned experiment. All 
\verb|c++|-codes are compiled with the C++20 standard (\verb|-std=c++20|) and an optimization level two (\verb|-O2|).

\subsection*{Simulation code}

The folder \verb|generateK/| includes the files \verb|main.cpp| with the main-function, \verb|system.hpp| with all used declarations and \verb|system.cpp|
contains the corresponding definitions. Before compiling, the lines 17-20 of the header file may have to modified for the desired purpose. In line 17 the object-like macro 
set to \verb|true| correspond to a compiling with the line spin update algorithm and to \verb|false| with the single spin update. The integer in 
line 18 correspond to the lattice length $L$. The coupling constants $J_1$ and $J_2$ are evaluated by the floating point numbers in lines 19 and 20.
A set of six command line arguments must be passed to the resulting executable.
\begin{itemize}
  \item[1.] PRNG seed as unsigned integer 
  \item[2.] simulation temperature $T$ as floating point number
  \item[3.] initial state $I$ as character \verb|F| correspond to the ferromagnetic initial state, \verb|H| to the horizontal striped one, \verb|V| to the
             vertical striped one and \verb|U| or any other character to the uniformly distributed one
  \item[4.] number of equilibration sweeps $N_\mathrm{equi}$
  \item[5.] number of sweeps for data collection  $N_\mathrm{meas}$
  \item[6.] number of sweeps within a subtracting block for the Jackknife method $m_J$
\end{itemize}
These arguments are chosen according to the Tables \ref{table:experiment_kalz1} and \ref{table:experiment_kalz2} for the considered lattice lengths. The compiled 
executable generates the folder
\verb|L_-_/| ($1^\text{st}$ gap correspond to $L$ and $2^\text{nd}$ one to the initial state) in the current working directory, if the folder does not already exist. In this folder a 
subfolder is generated corresponding to the chosen $T$. The entire energy histogram with the magnetization-lists (\verb|entire.txt|) is stored
in this folder as well as the $N_\mathrm{meas}/m_J$ block histograms (\verb|block_.txt|). Furthermore, the final PRNG engine state, the final lattice state, the final values 
of $E$, $M$, $M^2$,$M_s$, $M^2_s$ and the PRNG seed, $N_\mathrm{equi}$, $N_\mathrm{meas}$, $m_J$ is written in \verb|save.txt| for a possible continuing of
the simulation.  

\begin{table}[!h]
  \centering
  \begin{tabular}{|lllll|}
    \hline
    $L$  & $I$ & $N_\mathrm{equi}$ & $N_\mathrm{simu}$ & $N_\mathrm{jack}$ \\ 
    \hline
    $30$ & \verb|U| & $2.0\cdot10^{7}$ & $5.0\cdot10^{7}$ & $5.0\cdot10^{5}$\\
    $40$ & \verb|U| & $2.0\cdot10^{7}$ & $5.0\cdot10^{7}$ & $5.0\cdot10^{5}$\\
    $50$ & \verb|U| & $2.0\cdot10^{7}$ & $5.0\cdot10^{7}$ & $5.0\cdot10^{5}$\\
    $60$ & \verb|U| & $2.0\cdot10^{7}$ & $5.0\cdot10^{7}$ & $5.0\cdot10^{5}$\\
    $70$ & \verb|U| & $5.0\cdot10^{7}$ & $1.0\cdot10^{8}$ & $1.0\cdot10^{6}$\\
    $80$ & \verb|U| & $5.0\cdot10^{7}$ & $1.0\cdot10^{8}$ & $1.0\cdot10^{6}$\\
    $90$ & \verb|U| & $5.0\cdot10^{7}$ & $1.0\cdot10^{8}$ & $1.0\cdot10^{6}$\\
    $100$ & \verb|U| & $1.0\cdot10^{8}$ & $1.5\cdot10^{8}$ & $1.5\cdot10^{6}$\\
    $110$ & \verb|U| & $1.0\cdot10^{8}$ & $1.5\cdot10^{8}$ & $1.5\cdot10^{6}$\\
    $120$ & \verb|U| & $1.0\cdot10^{8}$ & $1.5\cdot10^{8}$ & $1.5\cdot10^{6}$\\
    \hline
  \end{tabular}
  \caption{This table shows all simulation parameters except for the temperatures and the PRNG seeds.}
  \label{table:experiment_kalz1}
\end{table}

\begin{table}[!h]
  \centering
    \begin{tabular}{|ll|}
      \hline
      $L$ & $(T,\text{seed})$ \\ 
      \hline
      $30$ & \makecell{$(0.37,9481)$,$(0.386,8879)$,$(0.402,1876)$,$(0.418,8270)$,\\$(0.434,1628)$,$(0.45,9600)$,$(0.466,5181)$,$(0.482,8255)$,\\$(0.498,7964)$,$(0.514,7966)$,$(0.53,8799)$} \\
      \hline
      $40$ & \makecell{$(0.35,3101)$,$(0.36,6454)$,$(0.37,2636)$,$(0.38,3449)$,\\$(0.39,4370)$,$(0.4,3573)$,$(0.41,2464)$,$(0.42,9733)$,\\$(0.43,6951)$,$(0.44,9248)$,$(0.45,404)$} \\
      \hline
      $50$ & \makecell{$(0.34,1262)$,$(0.35,3959)$,$(0.36,323)$,$(0.37,7862)$,\\$(0.38,3965)$,$(0.39,9907)$,$(0.4,4951)$,$(0.41,353)$,\\$(0.42,4764)$,$(0.43,6756)$,$(0.44,5063)$} \\
      \hline
      $60$ & \makecell{$(0.33,1304)$,$(0.34,9748)$,$(0.35,666)$,$(0.36,7767)$,\\$(0.37,6541)$,$(0.38,6140)$,$(0.39,4617)$,$(0.4,848)$,\\$(0.41,8426)$,$(0.42,8597)$,$(0.43,5430)$} \\
      \hline
      $70$ & \makecell{$(0.32,9334)$,$(0.33,5248)$,$(0.34,1710)$,$(0.35,5503)$,\\$(0.36,5083)$,$(0.37,4443)$,$(0.38,8094)$,$(0.39,8390)$,\\$(0.4,3062)$,$(0.41,4150)$,$(0.42,7934)$} \\
      \hline
      $80$ & \makecell{$(0.32,5660)$,$(0.326,478)$,$(0.332,286)$,$(0.338,5366)$,\\$(0.344,3288)$,$(0.35,4412)$,$(0.356,4696)$,$(0.362,4554)$,\\$(0.368,7866)$,$(0.374,4665)$,$(0.38,2323)$} \\
      \hline
      $90$ & \makecell{$(0.32,849)$,$(0.325,120)$,$(0.33,4593)$,$(0.335,8940)$,\\$(0.34,80)$,$(0.345,8962)$,$(0.35,4891)$,$(0.355,3008)$,\\$(0.36,115)$,$(0.365,4401)$,$(0.37,8335)$} \\
      \hline
      $100$ & \makecell{$(0.31,7564)$,$(0.315,545)$,$(0.32,7460)$,$(0.325,6423)$,\\$(0.33,819)$,$(0.335,6148)$,$(0.34,2233)$,$(0.345,7)$,\\$(0.35,9428)$,$(0.355,7516)$,$(0.36,4740)$} \\
      \hline
      $110$ & \makecell{$(0.31,9200)$,$(0.314,9894)$,$(0.319,5784)$,$(0.323,6224)$,\\$(0.328,3936)$,$(0.332,4971)$,$(0.337,6983)$,$(0.341,672)$,\\$(0.346,6819)$,$(0.35,6636)$} \\
      \hline
      $120$ & \makecell{$(0.3,7976)$,$(0.304,2828)$,$(0.309,9951)$,$(0.313,3972)$,\\$(0.318,2650)$,$(0.322,620)$,$(0.327,2653)$,$(0.331,3733)$,\\$(0.336,3299)$,$(0.34,1100)$} \\
      \hline
    \end{tabular}
  \caption{The chosen temperatures and PRNG seeds are depicted for the considered lattice length.}
  \label{table:experiment_kalz2}
\end{table}




\subsection*{Data analysis codes}

All codes for the data analysis are contained in the folder \verb|analyzeK/| in \verb|submission/|. The executable of \verb|mean.cpp| needs 
the following two command line arguments.
\begin{itemize}
  \item[1.] relative path from the current working directory to the target folder \verb|L_-_/| and the corresponding name
  \item[2.] relative path from the current working directory and name of the output \verb|txt|-file
\end{itemize}
This executable computes for all temperatures in \verb|L_-_/| the means and errors of the considered observables through the obtained histograms 
and magnetization lists. Compiling \verb|wham.cpp| results in an executable, generating reweighed data points of the considered 
observables as a function of $T$ with WHAM. The five command line arguments have to pass for a faultless execution.
\begin{itemize}
  \item[1.] relative path from the current working directory to the target folder \verb|L_-_/| and the corresponding name
  \item[2.] relative path from the current working directory and name of the output \verb|txt|-file
  \item[3.] the lower temperature boundary $T_\mathrm{min}$ as floating point number of the reweighing range
  \item[4.] the upper temperature boundary $T_\mathrm{max}$ as floating point number of the reweighing range
  \item[5.] the number of reweighed data points
\end{itemize}
The executable of \verb|peak.cpp| determines for each temperature in \verb|L_-_/| all the Jackknife histograms corresponding
to their block number. The WHAM procedure is only applied to the specific heat, to obtain the maximum of $c$ for each block number, displaying on
the console. Furthermore, the lattice length, the mean and the error of the peak location is written in the output file.
The following set of command line arguments must be passed to the executable.
\begin{itemize}
  \item[1.] relative path from the current working directory to the target folder \verb|L_-_/| and the corresponding name
  \item[2.] relative path from the current working directory and name of the output \verb|txt|-file
  \item[3.] number of Jackknife blocks
  \item[4.] the lower temperature boundary $T_\mathrm{min}$ as floating point number of the reweighing range
  \item[5.] the upper temperature boundary $T_\mathrm{max}$ as floating point number of the reweighing range
  \item[6.] the number of reweighed data points
\end{itemize}
The same output \verb|txt|-file should be chosen while applying this executable on different target folders (different $L$) or repeating the execution 
of a target, because the corresponding data will be appended or overwritten in this file.
The reweighing temperature ranges are chosen according to the smallest and largest considered temperatures in \verb|L_-_/|, further the number of the reweighed 
data points is selected such that a temperature precession of $0.000001$ is achieved. 



\section{Consistency checks}
\label{sec:kalz_checks}

The implementations mentioned in the previous two sections are qualitatively verified by the specific heat data for $L\!\in\!\{50,80,100\}$ and 
$J_2\!\in\!\{-0.3,-0.6\}$. The same procedure depicted in Section \ref{sec:kalz_experiment} is applied for these lattice lengths and NNN-interaction
constants. For all cases the system is initiated in a high temperature state (\verb|U|) and is equilibrated through 
$N_\mathrm{equi}\!=\!1.0\cdot10^{6}$ sweeps. Afterward, the entire energy histograms and the corresponding magnetization lists are obtained by 
$N_\mathrm{meas}\!=\!5.0\cdot10^{6}$ measurements. For the errors, the subtracting block length is chosen $m_J\!=\!5.0\cdot10^{4}$. The temperatures and 
PRNG seeds are selected according to Table \ref{table:experiment_consi}. The specific heat reference data is extracted from the work of Kalz et al. 
\cite{Kalz2008}. 

\begin{figure}[!h]
  \includegraphics{consi_g3.eps}
  \caption{This plot shows the means and corresponding errors (squares) of the free energy derivatives for $J_2\!=\!-0.3$.
           The full lines refer to the data obtained by WHAM.}
  \label{fig:consi_g3}
\end{figure}

The data of the free energy derivatives for $J_2\!=\!-0.3$ is shown in Figure \ref{fig:consi_g3}. The simulated specific heat data agrees with the reference
values, some deviations in proximity of the maximum are noticeable for $L\!=\!100$, perhaps originated by $N_\mathrm{meas}$ selected too small. The received peak 
locations are shown in Table \ref{table:heat_consi}. While lowering the temperature the depicted magnetization per lattice site increases and approaches $1$, 
emphasizing the ferromagnetic ordering of the low temperature phase for $J_2\!\in\!(-1/2,0]$ mentioned in Section \ref{sec:phase_transitions}. 

\begin{table}[!h]
  \centering
  \begin{tabular}{|lllll|}
  \hline
  & \multicolumn{2}{c}{$J_2\!=\!-0.3$} & \multicolumn{2}{c|}{$J_2\!=\!-0.6$} \\
  $L$ & $T_\mathrm{max}$ & $c_{V,\mathrm{max}}$ & $T_\mathrm{max}$ & $c_{V,\mathrm{max}}$ \\
  \hline
  $50$ & $1.26589(93)$ & $2.073(13)$ & $0.97413(18)$ & $9.105(71)$ \\
  $80$ & $1.2598(45)$ & $2.324(27)$ & $0.97310(21)$ & $14.55(21)$ \\
  $100$ & $1.26022(77)$ & $2.455(30)$ & $0.97242(26)$ & $18.70(48)$ \\
  \hline
  \end{tabular}
  \caption{The observed peak locations are shown in this table for the considered cases.}
  \label{table:heat_consi}
\end{table}

The $J_2\!=\!-0.6$ case is illustrated in Figure \ref{fig:consi_g6}. The values of the specific heat is in a good agreement with the reference data of Kalz et al.
Furthermore, the peak locations and errors are shown in Table \ref{table:heat_consi} as well. The depicted magnetization behaves according to
the prediction of a super-antiferromagnetic ordering in the low temperature phase for $J_2\!\in\!(g^*,-1/2)$.  

\begin{figure}[!h]
  \includegraphics{consi_g6.eps}
  \caption{The free energy derivatives are shown for $J_2\!=\!-0.6$. Particularly, the magnetic observables
           are depicted for the super-antiferromagnetic case.}
  \label{fig:consi_g6}
\end{figure}

Especially, the vicinity of the specific heat and susceptibility emphasizes phenomenological the kind of the phase transition for the considered 
$J_2$. The maximums narrow and grow slowly for $J_2\!=\!-0.3$, presuming a power divergence in the thermodynamic limit (Continuous phase transition), 
compared to the rapidly narrowing and growing of the peaks for $J_2\!=\!-0.6$, which is an evidence of a $\delta$-function like singularity in the 
thermodynamic limit (Discontinuous phase transition) \cite{Kalz2008}.


\begin{table}[!h]
  \centering
  \begin{tabular}{|ll|}
    \multicolumn{2}{l}{$J_2\!=\!-0.3$}\\
    \hline
    $L$ & $(T,\text{seed})$ \\ 
    \hline
    $50$ & \makecell{$(1.1,9445)$,$(1.13,3126)$,$(1.16,7097)$,$(1.19,4494)$,\\$(1.22,6778)$,$(1.25,3437)$,$(1.28,3416)$,$(1.31,9142)$,\\$(1.34,1193)$,$(1.37,6181)$,$(1.4,7924)$} \\
    \hline
    $80$ & \makecell{$(1.1,5089)$,$(1.13,1936)$,$(1.16,5560)$,$(1.19,7161)$,\\$(1.22,7240)$,$(1.25,7910)$,$(1.28,8669)$,$(1.31,2216)$,\\$(1.34,6146)$,$(1.37,8672)$,$(1.4,7208)$} \\
    \hline
    $100$ & \makecell{$(1.1,7987)$,$(1.13,3823)$,$(1.16,9466)$,$(1.19,1387)$,\\$(1.22,4255)$,$(1.25,5861)$,$(1.265,666)$,$(1.28,3554)$,\\$(1.31,736)$,$(1.34,9762)$,$(1.37,3708)$,$(1.4,7651)$} \\
    \hline
    \multicolumn{2}{l}{}\\
    \multicolumn{2}{l}{$J_2\!=\!-0.6$}\\
    \hline
    $L$ & $(T,\text{seed})$ \\ 
    \hline
    $50$ & \makecell{$(0.94,7880)$,$(0.95,5997)$,$(0.96,7716)$,$(0.97,746)$,\\$(0.98,3313)$,$(0.99,9276)$,$(1.0,247)$,$(1.01,5423)$,\\$(1.02,3587)$,$(1.03,3602)$,$(1.04,2052)$} \\
    \hline
    $80$ & \makecell{$(0.94,8849)$,$(0.95,1158)$,$(0.96,7295)$,$(0.97,8570)$,\\$(0.98,5735)$,$(0.99,20)$,$(1.0,2357)$,$(1.01,9621)$,\\$(1.02,4962)$,$(1.03,4354)$,$(1.04,5475)$} \\
    \hline
    $100$ & \makecell{$(0.94,6845)$,$(0.95,3487)$,$(0.96,7288)$,$(0.97,3574)$,\\$(0.98,9363)$,$(0.99,4849)$,$(1.0,5107)$,$(1.01,5831)$,\\$(1.02,3572)$,$(1.03,5922)$,$(1.04,5875)$} \\
    \hline
  \end{tabular}
  \caption{This table depicts the PRNG seed for the corresponding temperatures, lattice lengths and NNN-interaction constants.}
  \label{table:experiment_consi}
\end{table}






\chapter{Spin autocorrelation}

\section{Data generation details}

In analogy to the experiments of the free energy derivatives, the compiling and execution instruction are presented in the following for the experiments
in Section \ref{sec:timmons_experiment}. The used codes are in the subfolders \verb|generateT/| and \verb|analyzeT/| of the submission folder 
\verb|submission/|. The \verb|c++|-codes are compiled with the flags \verb|-std=c++20| and \verb|-O2|. The necessary \verb|python|-code requires 
the \verb|sys|, \verb|numpy| and \verb|scipy| modules. 

\subsection*{Simulation code}

The \verb|c++|-files associated with the simulation are located in the folder \verb|generateT/|. The file \verb|main.cpp| contains the main-function and
the callable (function pointer) for the thread objects. The remaining files \verb|system.hpp| and \verb|system.cpp| includes all declarations and definitions.
Perhaps, some lines of the header file has to be modified before compiling. The object like macro in line 19 set to \verb|true| enables the line spin update.
The integer and the both floating point numbers in line 20-22 correspond to $L$, $J_1$ and $J_2$. This implementation has the possibility of (trivial) multiple 
threading. The number of threads is chosen in line 23. The following command line arguments have to pass while starting the execution.
\begin{itemize}
  \item[1.] PRNG seed as unsigned integer
  \item[2.] lower boundary of considered temperature range $T_\mathrm{min}$ as floating point number
  \item[3.] upper boundary of considered temperature range $T_\mathrm{max}$ as floating point number
  \item[4.] step length between two temperatures $\Delta T$ as floating point number
  \item[5.] initial state $I$ as character \verb|F| correspond to the ferromagnetic initial state, \verb|H| to the horizontal striped one, \verb|V| to the
            vertical striped one and \verb|U| or any other character to the uniformly distributed one
  \item[6.] number of equilibration sweeps between two temperatures $N_\mathrm{chan}$
  \item[7.] number of sweeps for data collection $N_\mathrm{meas}$
  \item[8.] number of run repeations for the means and errors $N_\mathrm{repe}$
\end{itemize}
The chose arguments are depicted in Table \ref{table:experiment_timmons} for the different lattice length and update algorithms. The resulting executable
generates the folder \verb|L_-_/| corresponding to the chosen lattice length and initial state as well. In this folder the \verb|txt|-files \verb|T_.txt| 
are stored, containing the spin autocorrelation functions as function of $t_2$, associated to the simulated temperatures. Additionally, the
means and errors of the acceptance rate as function of $T$ are stored in \verb|acceptance.txt| for the line flip proposals and spin flip proposals.

\begin{table}[!h]
  \centering
  \begin{tabular}{|lllllllll|}
  \hline
  $L$ & seed & $T_\mathrm{min}$ & $T_\mathrm{max}$ & $\Delta T$ & $I$ & $N_\mathrm{chan}$ & $N_\mathrm{meas}$ & $N_\mathrm{repe}$ \\
  \hline
  $32$  & $712,3730$  & $0.2$  & $0.5$  & $0.01$ & \verb|F| & $1.0\!\cdot\!10^5$ & $1.0\!\cdot\!10^5$ & $50$ \\
  $64$  & $4323,9761$ & $0.2$  & $0.5$  & $0.01$ & \verb|F| & $1.0\!\cdot\!10^5$ & $1.0\!\cdot\!10^5$ & $50$ \\
  $128$ & $6577,1373$ & $0.2$  & $0.5$  & $0.01$ & \verb|F| & $1.0\!\cdot\!10^5$ & $1.0\!\cdot\!10^5$ & $50$ \\
        & $4116,6918$ & $0.3$  & $0.6$  & $0.01$ & \verb|U| & $5.0\!\cdot\!10^6$ & $1.0\!\cdot\!10^5$ & $4$ \\
  \hline
  \end{tabular}
  \caption{All the chosen simulation parameters are shown for both update algorithms. The first PRNG seed is associated with
           the single spin update and the second one to the line spin update.}
  \label{table:experiment_timmons}
\end{table}


\subsection*{Data analysis codes}

The \verb|python|-code for the data analysis is included in the folder \verb|analyzeT/|. The following parameters/arguments has be chosen 
in line 8-10 or has to be passed as command line arguments.
\begin{itemize}
  \item[1.] relative path from the current working directory to the target folder \verb|L_-_/| and the corresponding name
  \item[2.] lower boundary of considered temperature range $T_\mathrm{min}$ as floating point number 
  \item[3.] upper boundary of considered temperature range $T_\mathrm{max}$ as floating point number
\end{itemize}
The code generates in \verb|L_-_/| the \verb|txt|-file \verb|relax.txt| containing the relaxations times as a function of the temperature $T$
for the ferromagnetic, horizontal striped and vertical striped spin autocorrelation functions. Unfortunately, the code works only pretty well 
for the targets with initial states uniformly distributed (\verb|L_-U/|).  



