


A while ago, some controversies about spin glasses appeared in the condensed matter physics and statistical physics communities. Under discussion 
were the requirement of quenched disorder, i.e., an explicit dependence on randomness, or whether only frustration is urgently needed as it is for 
structural glasses. Therefore, similarities were presumed between these two kinds of glasses, turned out to be true for some systems. Further, a 
connection between structural glasses and regular magnetic system were closed.

A possible representative of such a system is the frustrated Ising model on a square lattice with competing nearest and next-nearest neighbor interactions.
This model is subjected to significant changes of the ground-state, depending on the ratio of the nearest and next-nearest neighbor interaction. 
The point where these two regimes collide covers a small range around a numerical value of the interaction ratio. In this thesis, this range will 
be denoted as point for simplicity. Further, these regimes exhibit different kinds of ground state patterns (ferromagnetic and super-antiferromagnetic) 
and different kinds of phase transition (continuous and discontinuous), both regimes become 
continuous with enlarging distance to the point of colliding~\cite{Kalz2008,Jin2012,Yoshiyama2023}. The nature of the low temperature phase at the colliding point remains an open question. However, 
it is known that at this point the system is subjected to a high frustration, i.e., it effectively freezes into a metastable state. Kalz et al. developed a 
simulation algorithm in particular for this case, to overcome the high frustration~\cite{Kalz2008}. Timmons et al. propose the high frustration is an 
evidence for glassy behavior~\cite{Timmons2018}. The focus of this thesis is to observe any glassy behavior with the mentioned simulation algorithm.  

This work is structured as follows. Firstly, the physical background is covered from the definition of the considered model and its properties to the 
statistical physics treatment, Chapter~\ref{cha:physics}. Chapter~\ref{cha:methods} contains all used methods for the computer simulation and the 
corresponding data analysis. After that, the procedure of data collection is explained and the associated data is presented and connected to the 
statistical physics properties, Chapter~\ref{cha:results}. Finally, the findings are summarized, and some outlooks are given in 
Chapter~\ref{cha:conclusion}. The Appendix contains the implementation detail, the compiling and execution instructions and some consistency checks. 


