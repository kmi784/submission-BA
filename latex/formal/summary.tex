\label{cha:conclusion}

\section{Summary}

For the equilibrium properties the results and the corresponding discussion of the simulated data lead to the following conclusion. 
The analysis of the specific heat suggests a peak at the temperature $T_\mathrm{max}\!=\!0$ in the thermodynamic limit, 
and hence the system exhibits no phase transition at a finite temperature. The exponential divergence presumption of the correlation 
length Ref.~\cite{Lee2024} could be further supported, Table~\ref{table:fits_heat}. 
There is no significant behavior of the magnetic observables findable within the considered temperature range.

The dynamical properties of the single spin update depicts highly correlated measurements at low temperatures, resulting 
from the frustrations. In comparison, the line spin update overcomes these frustrations and samples the low temperature range very well, 
Figure~\ref{fig:auto}. For the quenches into a high temperature state (warming run), a memory effect, being noticeable by the spin autocorrelations remaining in excess, 
could not be observed for the line spin updates, arguing against a glassy behavior, Figure~\ref{fig:auto}. The relaxation times of the low temperature
quenches (cooling run) have quite a high value but no singular behavior for the line spin update. This does not fully contradict the glassy behavior. Nevertheless, 
the findings suggests that the obtained behavior in Ref.~\cite{Timmons2018} is a dynamic effect of the used simulation algorithm. Therefore, 
the apparent glassiness results perhaps from high frustration, particularly in combination with the single spin update. However, this presumption has to be further 
verified by continuing studies.


\section{Outlook}

A possible continuing study is a variation of the system under consideration, in particular other boundary conditions e.g. free boundary conditions.
Binder et al.~\cite{Binder1986} mention that a spin glass behavior is highly subjected to boundary conditions. Thus, similar observations could 
support the glassiness. Another approach is to construct an order parameter in the fashion of Equation~\eqref{align:stripped_magnetization} with 
the other ground-states in addition to the ferromagnetic, vertical and horizontal striped ones. However, this could be quite cumbersome in implementations for 
larger system sizes, caused by a high degeneracy $2^{L+1}\!-\!1$, Section~\ref{sec:GroundState}. With this potential order parameter, studies of its 
susceptibility appear to be more fruitful than these one associated to a single representative of the ground state.

